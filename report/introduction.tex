\chapter{Introduction}
In our data-driven age, the ability to visualize complex datasets not only simplifies information but also tells a story, highlights patterns, trends, and anomalies that might otherwise remain hidden. Data visualization acts as the bridge between the raw, often impenetrable world of numbers and an accessible visual representation that can be easily understood, interpreted, and acted upon from any perspective.

In this report, we delve into the fundamentals of data visualization, elucidating its importance and methodologies. By applying these principles, we present a series of visualizations centered around wildfire data from California. Through these graphics, I aim to shed light on the frequency, intensity, and geographical spread of these fires over recent years.

Join me on this journey through the five learning objectives, as we navigate the terrain of data visualization and uncover the hidden tales of California's fiery landscape --- a landscape where not only trees and vegetation are at risk, but also a realm of danger for its myriad forest animals.

The data utilized in this report has been sourced from California's Department of Forestry and Fire Protection, available at \href{https://fire.ca.gov/}{www.fire.ca.gov}. This comprehensive dataset covers an eight-year span, from 2013 to 2020.

The project repository of this report can be found at \href{https://github.com/okaynils/fhnw-ds-gdv}{github.com/okaynils/fhnw-ds-gdv}.